\documentclass{exam}

\usepackage{float}
\usepackage[dutch]{babel}
\usepackage{hyperref}
\usepackage{graphicx}
\usepackage{cite}
\usepackage{url}
\usepackage{titlesec}
\usepackage{xcolor}
\usepackage{babelbib}
\usepackage[utf8]{inputenc}
\usepackage{url}
\usepackage{mathtools}


%code voor subsubsubsection te kunnen gebruiken
\titleclass{\subsubsubsection}{straight}[\subsection]

\newcounter{subsubsubsection}[subsubsection]
\renewcommand\thesubsubsubsection{\thesubsubsection.\arabic{subsubsubsection}}
\renewcommand\theparagraph{\thesubsubsubsection.\arabic{paragraph}}

\titleformat{\subsubsubsection}
  {\normalfont\normalsize\bfseries}{\thesubsubsubsection}{1em}{}
\titlespacing*{\subsubsubsection}
{0pt}{3.25ex plus 1ex minus .2ex}{1.5ex plus .2ex}

\makeatletter
\renewcommand\paragraph{\@startsection{paragraph}{5}{\z@}%
  {3.25ex \@plus1ex \@minus.2ex}%
  {-1em}%
  {\normalfont\normalsize\bfseries}}
\renewcommand\subparagraph{\@startsection{subparagraph}{6}{\parindent}%
  {3.25ex \@plus1ex \@minus .2ex}%
  {-1em}%
  {\normalfont\normalsize\bfseries}}
\def\toclevel@subsubsubsection{4}
\def\toclevel@paragraph{5}
\def\toclevel@paragraph{6}
\def\l@subsubsubsection{\@dottedtocline{4}{7em}{4em}}
\def\l@paragraph{\@dottedtocline{5}{10em}{5em}}
\def\l@subparagraph{\@dottedtocline{6}{14em}{6em}}
\makeatother

\setcounter{secnumdepth}{4}
\setcounter{tocdepth}{4}


% voor java code
\usepackage{listings}
\usepackage{color}

\definecolor{dkgreen}{rgb}{0,0.6,0}
\definecolor{gray}{rgb}{0.5,0.5,0.5}
\definecolor{mauve}{rgb}{0.58,0,0.82}

\lstset{frame=tb,
  language=Java,
  aboveskip=3mm,
  belowskip=3mm,
  showstringspaces=false,
  columns=flexible,
  basicstyle={\small\ttfamily},
  numbers=none,
  numberstyle=\tiny\color{gray},
  keywordstyle=\color{blue},
  commentstyle=\color{dkgreen},
  stringstyle=\color{mauve},
  breaklines=true,
  breakatwhitespace=true,
  tabsize=3
}


%template voorpagina
\newcommand*{\titleGP}{\begingroup % Create the command for including the title page in the document
\centering % Center all text
\vspace*{\baselineskip} % White space at the top of the page

\rule{\textwidth}{1.6pt}\vspace*{-\baselineskip}\vspace*{2pt} % Thick horizontal line
\rule{\textwidth}{0.4pt}\\[\baselineskip] % Thin horizontal line

{\LARGE PDS: Eindverslag convolutie}\\[0.2\baselineskip] % Title

\rule{\textwidth}{0.4pt}\vspace*{-\baselineskip}\vspace{3.2pt} % Thin horizontal line
\rule{\textwidth}{1.6pt}\\[\baselineskip] % Thick horizontal line

\scshape % Small caps
2016--2017\par

\vspace*{2\baselineskip}

{\scshape Jens Vannitsen (1334039) \vspace*{2\baselineskip}\par}

\vfill

\endgroup}

\begin{document}

\begin{questions}
\question \textbf{Exercise 5.1.1:  }Compute the PageRank of each page in Fig. 5.7, assuming no taxation.\\


    \begin{tabular}{ | l | l | l | l |}
    \hline
    & \textbf{a} & \textbf{b} & \textbf{c} \\ \hline
    \textbf{a} & 1/3 & 1/2 & 0\\ \hline
    \textbf{b} & 1/3 & 0 & 1/2 \\ \hline
    \textbf{c} & 1/3 & 1/2 & 1/2  \\
    \hline
    \end{tabular}


$ r = \begin{bmatrix}
       \frac{1}{3} \\
       \frac{1}{3}\\
       \frac{1}{3}      
     \end{bmatrix}$

Na T iteraties $r * M$ te berekenen krijgen we $\lambda = \begin{bmatrix}
       \frac{3}{13} \\ \frac{4}{13} \\   \frac{6}{13}
     \end{bmatrix}$
 \\
 of\\
 
$ \begin{bmatrix}
       \frac{1}{x} \\
       \frac{1}{y}\\
       \frac{1}{z}      
     \end{bmatrix} = M * \begin{bmatrix}
       \frac{1}{x} \\
       \frac{1}{y}\\
       \frac{1}{z}      
     \end{bmatrix} $    \\\\
     
     $x = 1/3x + 1/2y $\\
     $y = 1/3x + 1/2z $\\
     $z = 1/3 + 1/2y + 1/2Z$\\
     $x + y + z = 1$

\question \textbf{Exercise 5.1.2:} Compute the PageRank of each page in Fig. 5.7, assuming
$\beta = 0.8$.\\
opmerking: Deze methode is de uitwerking uitgelegd in de slides, in het boek wordt een andere methode uitgelegd.\\
$A = \beta * M + (1 - \beta) * \begin{bmatrix}\frac{1}{N}\end{bmatrix}_{NxN}$

$A = 0.8 *  \begin{bmatrix}1/3 & 1/2 & 0\\ 
 1/3 & 0 & 1/2 \\ 
 1/3 & 1/2 & 1/2 \end{bmatrix} + 0.2 *  \begin{bmatrix}1/3 & 1/3 & 1/3\\ 
1/3 & 1/3 & 1/3\\ 
1/3 & 1/3 & 1/3 \end{bmatrix}$\\

$ r = \begin{bmatrix}
       \frac{1}{3} \\
       \frac{1}{3}\\
       \frac{1}{3}      
     \end{bmatrix}$\\
We voeren vervolgens T keer r * A uit en krijgen  $\lambda = \begin{bmatrix}
       \frac{7}{27} \\ \frac{25}{81} \\   \frac{35}{81}
     \end{bmatrix}$\\

\question \textbf{Exercise 5.1.3:} Suppose the Web consists of a clique (set of nodes with all possible arcs from one to another) of n nodes and a single additional node that is the successor of each of the n nodes in the clique. Figure 5.8 shows this graph for the case n = 4. Determine the PageRank of each page, as a function of n and $\beta$.     \\
     
Voor iedere node geldt dat deze een connectie heeft naar de andere nodes met een waarde van $\frac{1}{N}$ behalve de successor node, deze heeft in de matrix een kolom die enkel uit nullen bestaat.
voor bijvoorbeeld n = 4 zoals figuur 5.8 krijgen we:\\\\
$ M = \begin{bmatrix}0 & 1/n & 1/n & 1/n & 0\\ 
1/n & 0 & 1/n & 1/n & 0\\ 
1/n & 1/n & 0 & 1/n & 0\\ 
1/n & 1/n & 1/n & 0& 0\\ 
1/n & 1/n & 1/n & 1/n & 0 \end{bmatrix}$\\

$\begin{bmatrix}
       \frac{1}{x_{1}} \\
       \frac{1}{x_{2}}\\
       \frac{1}{x_{3}}\\
        \frac{1}{x_{4}}\\     
     \end{bmatrix}  = \beta * M * \begin{bmatrix}
       \frac{1}{x_{1}} \\
       \frac{1}{x_{2}}\\
       \frac{1}{x_{3}}\\
        \frac{1}{x_{4}}\\     
     \end{bmatrix} + (1 - \beta) * \begin{bmatrix}
       \frac{1}{1/n+1} \\
       \frac{1}{1/n+1}\\
       \frac{1}{1/n+1}\\
        \frac{1}{1/n+1}\\     
     \end{bmatrix}$ \\
     
     $x_{1} = \dots \\
     x_{2} = \dots \\
     x_{3} = \dots \\
     x_{1} + x_{2} + x_{3} + x_{4} = 1$


\question \textbf{Exercise 5.2.2 (a):} Using the method of Section 5.2.1, represent the transition matrices of the following graphs: (a) Figure 5.4.  \\ 

\begin{tabular}{ | l | l | l | l |}
    \hline
    \textbf{source} & \textbf{degree} & \textbf{destinations} \\ \hline
    \textbf{A} & 3 & B,C,D \\ \hline
    \textbf{B} & 2 &  A,D \\ \hline
    \textbf{C} & 1 & E \\ \hline
      \textbf{D} & 2 & B,C  \\
    \hline
    \end{tabular}
    
\question \textbf{Exercise 5.3.1 : } Compute the topic-sensitive PageRank for the graph of Fig. 5.15, assuming the teleport set is: ($\beta = 0.8$)\\ \\
$\begin{bmatrix}0 & 1/2 & 1 & 0 \\ 
1/3 & 0 & 0 & 1/2 \\ 
1/3 & 0 & 0 & 1/2 \\ 
1/3 & 1/2 & 0 & 0 \end{bmatrix}$\\\\\\\
(a) A only:\\
$0.8 *  \begin{bmatrix}0 & 1/2 & 1 & 0 \\ 
1/3 & 0 & 0 & 1/2 \\ 
1/3 & 0 & 0 & 1/2 \\ 
1/3 & 1/2 & 0 & 0 \end{bmatrix} + 0.2*\begin{bmatrix} 1 \\ 
0 \\ 
0 \\
0 \end{bmatrix}$\\\\
start T iteraties beginnende van $\begin{bmatrix} 1 \\ 
0 \\ 
0 \\
0 \end{bmatrix}$ geeft ons v = $\begin{bmatrix} 3/7 \\ 4/21 \\ 4/21 \\ 4/21 \end{bmatrix}$\\\\
(b) A and C\\
$0.8 *  \begin{bmatrix}0 & 1/2 & 1 & 0 \\ 
1/3 & 0 & 0 & 1/2 \\ 
1/3 & 0 & 0 & 1/2 \\ 
1/3 & 1/2 & 0 & 0 \end{bmatrix} + 0.2*\begin{bmatrix} 1/2 \\ 
0 \\ 
1/2 \\
0 \end{bmatrix}$\\\\
start T iteraties beginnende van $\begin{bmatrix} 1/2 \\ 
0 \\ 
1/2 \\
0 \end{bmatrix}$ geeft ons v = $\begin{bmatrix}  19/50 \\ 17/100 \\ 27/100 \\ 17/100 \end{bmatrix}$\\\\



\question \textbf{Exercise 5.4.1 :} In Section 5.4.2 we analyzed the spam farm of Fig. 5.16, where every supporting page links back to the target page. Repeat the analysis for a spam farm in which:\\\\
opmerking: moet opgeschreven worden zoals de analyse uit het boek.\\
(a) Each supporting page links to itself instead of to the target page. \\\\
pagerank van de target node gaat fel zakken omdat deze niet wordt terug gestuurd.

(b) Each supporting page links nowhere.\\\\

(c) Each supporting page links both to itself and to the target page.\\\\
pagerank wordt verdeeld tussen de supporting pages en de target page.

\question \textbf{Exercise 5.4.2 :} For the original Web graph of Fig. 5.1, assuming only B is a trusted page:\\\\
(a) Compute the TrustRank of each page.\\\\

$0.8 *  \begin{bmatrix}0 & 1/2 & 1 & 0 \\ 
1/3 & 0 & 0 & 1/2 \\ 
1/3 & 0 & 0 & 1/2 \\ 
1/3 & 1/2 & 0 & 0 \end{bmatrix} + 0.2*\begin{bmatrix} 1/2 \\ 
0 \\ 
1/2 \\
0 \end{bmatrix}$\\\\
start T iteraties beginnende van $\begin{bmatrix} 0 \\ 
1 \\ 
0 \\
0 \end{bmatrix}$ geeft ons v = $\begin{bmatrix}  27/100 \\ 9/25 \\ 4/25 \\ 21/100 \end{bmatrix}$\\\\

(b) Compute the spam mass of each page.\\\\

$0.8 *  \begin{bmatrix}0 & 1/2 & 1 & 0 \\ 
1/3 & 0 & 0 & 1/2 \\ 
1/3 & 0 & 0 & 1/2 \\ 
1/3 & 1/2 & 0 & 0 \end{bmatrix} + 0.2* \begin{bmatrix}1/4 & 1/4 & 1/4 & 1/4 \\ 
1/4 & 1/4 & 1/4 & 1/4 \\ 
1/4 & 1/4 & 1/4 & 1/4 \\ 
1/4 & 1/4 & 1/4 & 1/4 \end{bmatrix}$\\\\
T iteraties geeft ons v = $\begin{bmatrix}  3/9 \\ 2/9 \\ 2/9\\ 2/9\end{bmatrix}$\\\\

spam mass formule = $\frac{pagerank - trustrank}{pagerank}$

    \begin{tabular}{ | l | l | }
    \hline
    \textbf{node} & \textbf{spam mass} \\ \hline
    \textbf{a} & 0.19 \\ \hline
    \textbf{b} & -0.62\\ \hline
    \textbf{c} &  0.28 \\ \hline
    \textbf{d} &  0.055 \\
    \hline
    \end{tabular}



\question \textbf{Exercise 5.5.1 :} Compute the hubbiness and authority of each of the nodes in our original Web graph of Fig. 5.1.

$L = \begin{bmatrix}0 & 1 & 1 & 0 \\ 
1 & 0 & 0 & 1 \\ 
1 & 0 & 0 & 1 \\ 
1 & 1 & 0 & 0 \end{bmatrix}, L^T = \begin{bmatrix}0 & 1 & 1 & 1 \\ 
1 & 0 & 0 & 1 \\ 
1 & 0 & 0 & 0 \\ 
0 & 1 & 1 & 0 \end{bmatrix}$\\\\

in de slides staat dat we h en a moeten initialiseren op $1/\sqrt{N}$, maar in het boek staat initialiseer op 1. Na het berekenen van h en a moet men steeds normaliseren, de maximum waarde in de kolom is 1.\\
$h = La$\\
$a = L^Th$\\\\

$h = \begin{bmatrix}0,2891  \\ 
1  \\ 
1  \\ 
0.8136  \end{bmatrix}$\\\\\\\\\
$a = \begin{bmatrix}1  \\ 
0.3919  \\ 
0.1027  \\ 
0.7108  \end{bmatrix}$\\
\end{questions}
\end{document}